\documentclass[10pt,a4paper]{article}
\usepackage[latin1]{inputenc}
\usepackage[margin=1in]{geometry}
\usepackage{amsmath}
\usepackage{amsfonts}
\usepackage{amssymb}
\usepackage{graphicx}
\usepackage{comment}
\begin{document}
	
\section{Data Acquisition}

\begin{table}[h!]
		\centering
		\begin{tabular}{l l l}
			r1c1 & r1c2 & r1c3 \\
			r2c1 & r2c2 & r2c3 \\
		\end{tabular}
		\caption{Subsystem Specifications}
		\label{tab:example}
\end{table}

\subsection{Design}

\subsubsection{Data Requirements}

Discuss the data needs: thrust, tank pressure, chamber pressure, ambient temperature, chamber (x2) temperature, and tank temperature. Show a diagram of the rocket engine in the test stand with the location of each sensor.

\begin{figure}[h!]
	\centering
	\includegraphics[width=0.25\textwidth]{./figs/logo_srt.png}
	\caption{System Diagram}
	\label{fig:example}
\end{figure}
	
\subsubsection{Hardware Discussion}

\paragraph{Load Cell}

Discuss the load cell type, the precision, output voltage, and operating conditions. Discuss the inputs/outputs. Discuss the need for a signal conditioner and how it's used. 

\paragraph{Pressure Transducer}

Discuss the PT types, precision, and inputs/outputs. 

\paragraph{Thermocouple}

Discuss the TC types, precision, and inputs/outputs. Discuss why K-type is used. Discuss the need for analog amplifiers and the type used. 

\paragraph{DAQ Board}

Discuss the DAQ board used (NI 6341) and its specifications. Discuss how each sensor is connector to the DAQ board. Quantify the ADC precision for each sensor. 
	
\subsubsection{Software Discussion}

Discuss the Labview SRT DAQ program: its function, input/outputs, and a view of the front panel.  

\begin{figure}[h!]
	\centering
	\includegraphics[width=0.25\textwidth]{./figs/logo_srt.png}
	\caption{Font Panel Screenshot}
	\label{fig:example}
\end{figure}

\newpage
\subsection{Manufacturing}

\subsubsection{Circuit Diagram}

Show a circuit diagram of the DAQ system.

\begin{figure}[h!]
	\centering
	\includegraphics[width=0.25\textwidth]{./figs/logo_srt.png}
	\caption{Circuit Diagram}
	\label{fig:example}
\end{figure}

\subsubsection{Physical Schematic}

Show a physical schematic of the DAQ system in the junction box. 

\begin{figure}[h!]
	\centering
	\includegraphics[width=0.25\textwidth]{./figs/logo_srt.png}
	\caption{Physical Schematic}
	\label{fig:example}
\end{figure}

\subsubsection{Communication Schematic}

Show a schematic of the DAQ system communication (Control Room --- External Computers --- DAQ Boards --- Junction Box --- Sensors)

\begin{figure}[h!]
	\centering
	\includegraphics[width=0.25\textwidth]{./figs/logo_srt.png}
	\caption{Communication Schematic}
	\label{fig:example}
\end{figure}

\newpage
\subsection{Testing}

\subsubsection{Noise \& Filters}

\paragraph{System Noise}

Characterize the system noise and its source(s). This should include an FFT.

\paragraph{Filter Design}

Design an analog filter for the Thermocouples. 

Discuss digital filters for the Load Cell and Pressure Transducer data. 

\subsubsection{Sensor Calibration}

\paragraph{Load Cell}

Discuss load cell calibration, with regards to Gain and Balance. Show calibration curve(s). 

\paragraph{Pressure Transducer}

Discuss Pressure Transducer calibration technique(s) and results.

\paragraph{Therocmouple}

Discuss Thermocouple calibration technique(s) and results.

\begin{thebibliography}{10}
	
	\bibitem{example}
	Fernandez, M.M., 
	"Propellent Tank Pressurization Modeling for a Hybrid Rocket,"
	Thesis, Mechanical Engineering Dept., Rochester Institute of      
	Technology, 
	Rochester, NY, 2009.
	
\end{thebibliography}

\end{document}