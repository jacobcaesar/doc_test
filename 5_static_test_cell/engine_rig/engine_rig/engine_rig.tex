\documentclass[10pt,a4paper]{article}
\usepackage[latin1]{inputenc}
\usepackage[margin=1in]{geometry}
\usepackage{amsmath}
\usepackage{amsfonts}
\usepackage{amssymb}
\usepackage{graphicx}
\begin{document}
	
\section{Static Test Cell}

\begin{figure}[h!]
	\centering
	\includegraphics[width=0.25\textwidth]{./figs/logo_srt.png}
	\caption{Example}
	\label{fig:example}
\end{figure}


\begin{table}[h!]
		\centering
		\begin{tabular}{l l l}
			Transducer Techniques Load Cell & 500 lbs &  \\
			What else should be in RTC mech specs??? & r2c2 & r2c3 \\
		\end{tabular}
		\caption{Static Test Cell Specifications}
		\label{tab:example}
\end{table}
\subsection{Design}

 \begin{enumerate}
 	\item Brief overview of old system
 	\begin{enumerate}
 		\item General idea of how system works: how engine is mounted, where thrust is being measured, data available to be collected
 		\begin{enumerate}
 			\item Engine is mounted at combustion chamber flange to rig.
 			\item Rig is essentially a cage that is constrained by rollers within I-beams. Plate at top of rig has a rigid connection to load cell, and thrust is measured from the top of the set-up.
 		\end{enumerate}
 		\item Problems with system that drove redesign
 		\begin{enumerate}
 			\item LOAD CELL: Load cell data not depicting accurate thrust or thrust calculated from pressure transducers and thermocouples. Friction in the system; loss of peak thrust due to wood between load cell and rig (to protect against initial spike when engine ignites); potential off axis loads that would not register at load cell **OLD LOAD CELL DATA**
 			\item RIG: Alignment issues with rig pieces
 			\item ROLLERS: Tedious and difficult to align; caused friction along I-beams
 		\end{enumerate}
 	\end{enumerate}
 	
 	\item Redesign
 	\begin{enumerate}
 		\item Ball joint assembly to make set-up more robust and less susceptible to data loss
 		\begin{enumerate}
 			\item Ball joint accounts for any off axis forces due to it's ability to deflect by small angles
 			\item Replaced wood piece with steel; less likely to dampen initial peak thrust at beginning of ignition
 			\item **NEW LOAD CELL DATA**
 		\end{enumerate}
 		\item Rig redesign to make system set-up quicker, safer, simpler, and more robust
 		\begin{enumerate}
 			\item Switch angle iron configuration to provide easier access to plumbing and connections
 			\item Redesign top plate due to difficulty bolting angle irons while holding engine in place
 			\item New attachment points at combustion chamber flange pieces to prevent future alignment issues 
 			\item Replace rollers with simple pieces of conduit that will guide the rig (keep away from I-beams) but will not be a sources of friction due to constraints within I-beam **this probably needs a picture of where the friction is thought to occur - side of I beams - and how new roller will eliminate that**
 			\item modes of failure; analysis for each mode. Angle iron attachments, engine attachments, ball joint, etc.
 		\end{enumerate}
 	\end{enumerate}
 	\item Final specifications
 	\begin{enumerate}
 		\item FEA on ball joint assembly (one with force normal to ball joint face, one with force at small angle)
 		\item CAD of ball joint assembly (ref engr drawing)
 		\item CAD of over all rig with engine
 		\item CAD of close up on change of rollers (old and new)
 	\end{enumerate}
 \end{enumerate}

\newpage
\subsection{Manufacturing}

Here is where you will describe the manufacturing process of actually fabricating the part.
\begin{itemize}
	\item Ball Joint assembly pictures
	\item rig pics
	\item test cell pics 
\end{itemize}

\subsection{Testing}

Load cell data
\begin{itemize}
	\item old calibration curves
	\item new calibration curves
	\item compare thrust curves; peak thrust, fill weight, etc.
\end{itemize}


\begin{thebibliography}{10}
	
	\bibitem{example}
	your mom
	
\end{thebibliography}

\end{document}